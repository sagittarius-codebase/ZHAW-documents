\documentclass[a4paper,10pt]{article}
%\usepackage[utf8]{inputenc}
%\usepackage[T1]{fontenc}
\usepackage{xcolor}
\usepackage{amsmath, amssymb}
%\usepackage{tcolorbox}
\usepackage{../../../assets/LaTeX/style} % Adjust to your path

\title{Box Examples}
\author{Moritz Feuchter}
\date{\today}

\begin{document}
\maketitle

% 1. Lemma
\begin{lemma}{1. Example Lemma Title}
    This is an example lemma box. It can contain definitions or mathematical statements.\\
\end{lemma}

% 2: terms
\begin{terms}{2. Example Term Title}
    This is an example term box. You can use it to explain key terms or concepts.\\
\end{terms}

% 3. Definition
\begin{definition}{3. Example Definition Title}
    This is an example definition box. You can use it to define terms or concepts.\\
\end{definition}

% 4. Theorem
\begin{theorem}{4. Example Theorem Title}
    This is an example theorem box. You can state mathematical theorems or rules here.\\
\end{theorem}

% 5. Corollary
\begin{corollary}{5. Example Corollary Title}
    This is an example corollary box, typically used to state results derived from a theorem.\\
\end{corollary}

% 6. Formula
\begin{formula}{6. Example Formula Title}
    \[ E = mc^2 \]
    This box is useful for highlighting formulas or equations.\\
\end{formula}

% 7. Definition
\begin{definition}{7. Example Definition Title}
    A derivative of a function represents the rate of change of the function as the input changes.\\
\end{definition}

% 8. Concept
\begin{concept}{8. Example Concept Title}
    This is an example concept box. You can explain key concepts or ideas here.\\
\end{concept}

% 9. KR (How-to)
\begin{KR}{9. Example How-To Title}
    This is a how-to box, useful for explaining step-by-step instructions.\\
\end{KR}

% 10. Example
\begin{example}
    10. This is a simple example box. You can use it to showcase examples in your document.\\
\end{example}

% 11. Example2
\begin{example2}{11. Example Title}
    This is another version of the example box with a title.\\
\end{example2}

% 12. Remark
\begin{remark}
    12. This is a remark box for adding extra notes or side comments.\\
\end{remark}

% 13. Remark2
\begin{remark2}{13. Example Remark Title}
    This is another remark box with a title, used for detailed notes or warnings.\\
\end{remark2}

% 14. Highlight Box
\begin{highlight}{14. Highlighted Text}
    This is a highlighted box, perfect for drawing attention to important points or formulas.\\
\end{highlight}

% 15. Math Highlight
\[
    15. E = mc^2
\]
This is an inline math highlight box, useful for highlighting specific formulas.\\

% 16. Inline Equation
This is an inline equation with emphasis:
\begin{iequation}
    16. E = mc^2
\end{iequation}
\\
\end{document}
