\section{Einführung}\label{sec:Einführung}

\begin{definition}{Definition - Algorithmus\\}
    Ein Algorithmus ist eine endliche Folge von Anweisungen, die eine Eingabe in eine Ausgabe umwandelt.
\end{definition}

-> Algorithmisches Problem: Problem, das mit einem Algorithmus gelöst werden kann.
\\

\begin{concept}{Eigenschaften von Algorithmen}
    \\
    \begin{itemize}
        \item \textbf{Determiniertheit:} Identische Eingaben führen stets zu identischen Ergebnissen.
        \item \textbf{Determinismus:} Ablauf des Verfahrens ist an jedem Punkt fest vorgeschrieben (keine Wahlfreiheit).
        \item \textbf{Terminierung:} Für jede Eingabe liegt das Ergebnis nach endlich vielen Schritten vor.
        \item \textbf{Effizienz:} «Wirtschaftlichkeit» des Aufwands relativ zu einem vorgegebenen Massstab (z.B. Laufzeit, Speicherplatzverbrauch).
    \end{itemize}
    -> Diese Eigenschaften sind nicht immer erfüllt, aber wünschenswert.
\end{concept}



